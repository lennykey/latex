\documentclass{beamer}
\usepackage[utf8x]{inputenc}
%\usepackage[latin1]{inputenc}
\usepackage[ngerman]{babel}
\usepackage{graphicx}
\usetheme{Warsaw}
\usecolortheme{default}
\useinnertheme{circles}
%\useoutertheme{smoothbars}
%\useoutertheme{infolines}
%\useoutertheme{miniframes}
%\useoutertheme{sidebar}
%\useoutertheme{split}
%\useoutertheme{shadow}
%\useoutertheme{smoothtree}
%\useoutertheme{tree}
%\useoutertheme{miniframes}
\usepackage{amsmath}
\usepackage{amsfonts}
\usepackage{amssymb}
\setbeamercovered{transparent}
\title{WM Tippspiel 2010}
\author{Schächterle - Heimstädt - Cuartas}
\institute[]{Hochschule Augsburg}
%\logo{\pgfimage[width=2cm,height=2cm]{bilder/logo.jpg}}
\date{\today}

\begin{document}

\frame{\titlepage}
\section[]{}
\frame{
	\frametitle{Inhaltsverzeichnis}
	\begin{columns}[c]
	\begin{column}{5cm}
		\tableofcontents[section, subsectionstyle=hide]
	\end{column}
	\begin{column}{5cm}
		\begin{figure}[htbp]
 		 \centering
			\includegraphics[width=5cm,height=4cm]{images/wmtippspiel.png}
		\end{figure}
	\end{column}
	\end{columns}
}

\section{Aufgabenstellung}
\frame{
	\frametitle{Beschreibung}
		\begin{columns}[c]
		\begin{column}{5cm}
        \begin{block}{Beschreibung}
			\begin{itemize}
				\item Fußball Tippspiel   
				\item Tipper sollen auf die Spiele der WM 2010 in Südafrika tippen   
				\item Auswertung der erzielten Punkte    
				\item Vergleich mit anderen Spielern    
				\item Einfache Benutzung    
            \end{itemize}
        \end{block}
		\end{column}
		\begin{column}{5cm}
			\begin{figure}[htbp]
 			 \centering
				\includegraphics[width=4cm,height=3cm]{images/wmtippspiel.png}
			  \caption{WM Tippspiel}
			  \label{wmippspiel}
			\end{figure}

		\end{column}

		\end{columns}
}



\section{Technologien}
\frame{
	\frametitle{Django}
		\begin{columns}[c]
		\begin{column}{5cm}
        \begin{block}{Beschreibung}
			\begin{itemize}
				\item Nutzt MVC  
				\item OR-Mapper  
				\item URL-Dispatcher  
				\item Template Sprache  
            \end{itemize}
        \end{block}
		\end{column}
		\begin{column}{5cm}
			\begin{figure}[htbp]
 			 \centering
				\includegraphics[width=4cm,height=3cm]{images/django.jpg}
			  \caption{django}
			  \label{Django}
			\end{figure}

		\end{column}

		\end{columns}
}

\frame{
	\frametitle{Webserver}
		\begin{columns}[c]
		\begin{column}{5cm}
        \begin{block}{Beschreibung}
			\begin{itemize}
				\item Apache  
				\item mod wsgi (Verbindung zu Django) 
				\item PostgreSQL (Produktiv) 
				\item MySQL (Entwicklung) 
            \end{itemize}
        \end{block}
		\end{column}
		\begin{column}{5cm}
			\begin{figure}[htbp]
 			 \centering
				\includegraphics[width=4cm,height=3cm]{images/welt.png}
			  \caption{Web}
			  \label{web}
			\end{figure}

		\end{column}

		\end{columns}
}


\frame{
	\frametitle{User-Authentifikation}
		\begin{columns}[c]
		\begin{column}{5cm}
        \begin{block}{Django bietet an:}
			\begin{itemize}
				\item Technik: Session
				\item Models: User, Group, Permission
				\item Hilfsfunktionen (Shortcuts,..)
            \end{itemize}
        \end{block}
        \begin{block}{Entwickler macht:}
			\begin{itemize}
				\item Templates für Login
				\item Registrierung
				\item Passwort per Mail zuschicken
            \end{itemize}
        \end{block}
		\end{column}
		\begin{column}{5cm}
			\begin{figure}[htbp]
 			 \centering
				\includegraphics[width=4cm,height=4cm]{images/puzzle_lock.jpg}
			  \caption{User-Authentifikation}
			  \label{userauth}
			\end{figure}

		\end{column}

		\end{columns}
}

\frame{
	\frametitle{Weboberfläche}
		\begin{columns}[c]
		\begin{column}{5cm}
        \begin{block}{Geschäftslogik}
			\begin{itemize}
				\item Startseite bietet Feed an
				\item Tippen unterstützt mit Ajax, jQuery
				\item Vergleich/Highscore
            \end{itemize}
        \end{block}
		\end{column}
		\begin{column}{5cm}
			\begin{figure}[htbp]
 			 \centering
				\includegraphics[width=4cm,height=4cm]{images/zahnrad.jpg}
			  \caption{Weboberfläche}
			  \label{weboberflaeche}
			\end{figure}

		\end{column}

		\end{columns}
}

\frame{
	\frametitle{BeautifulSoup}
		\begin{columns}[c]
		\begin{column}{5cm}
        \begin{block}{HTML-Parsing}
			\begin{itemize}
				\item Mannschaften, Gruppen, Begegnungen und Ergebnisse der Spiele werden automatisch ausgelesen und importiert
				\item Regelmäßges Ausführen nach den Spielen
				\item Keine manuelle Eingabe der Daten nötig
				\item Quelle: de.Fifa.com
            \end{itemize}
        \end{block}
		\end{column}
		\begin{column}{5cm}
			\begin{figure}[htbp]
 			 \centering
				\includegraphics[width=4cm,height=4cm]{images/beautifulsoup.jpg}
			  \caption{BeautifulSoup}
			  \label{beautifulsoup}
			\end{figure}

		\end{column}

		\end{columns}
}

\frame{
	\frametitle{Universal Feed Parser}
		\begin{columns}[c]
		\begin{column}{5cm}
        \begin{block}{RSS-Parsing}
			\begin{itemize}
				\item Liest den News-Feed der Fifa aus
				\item Extrahiert Informationen wie Bilder, Titel, Inhalt, Datum, Link, usw.
				\item Werden auf der Startseite aufbereitet dargestellt
            \end{itemize}
        \end{block}
		\end{column}
		\begin{column}{5cm}
			\begin{figure}[htbp]
 			 \centering
				\includegraphics[width=4cm,height=4cm]{images/rss.jpg}
			  \caption{rss}
			  \label{rss}
			\end{figure}

		\end{column}

		\end{columns}
}


%\section{Implementierung}

%\frame{
%	\frametitle{Blah}
%		\begin{columns}[c]
%		\begin{column}{5cm}
%        \begin{block}{Beschreibung}
%			\begin{itemize}
%				\item Blah 
%            \end{itemize}
%        \end{block}
%		\end{column}
%		\begin{column}{5cm}
%			\begin{figure}[htbp]
% 			 \centering
%				\includegraphics[width=4cm,height=3cm]{images/welt.png}
%			  \caption{Container}
%			  \label{container}
%			\end{figure}
%
%		\end{column}
%		\end{columns}
%}

%\frame{
%
%	\frametitle{blah2}\begin{figure}[htbp]
% 			 \centering
%				\includegraphics[width=8cm,height=6cm]{images/welt.png}
%			  \caption{Blah2}
%			  \label{blah2}
%			\end{figure}
%}


\section{Demo}

\frame{
	\frametitle{Demo}\begin{figure}[htbp]
 			 \centering
				\includegraphics[width=10cm,height=4cm]{images/wmtippspiel.png}
			  \caption{Demo}
			  \label{demo}
			\end{figure}
}


\end{document}
