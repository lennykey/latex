% Ausarbeitung SAJ 
% FH Augsburg 
%
%
%
%
%\documentclass[titlepage, 12pt,a4paper]{scrartcl}
%scrartcl
%\usepackage[ngerman]{babel}
%%\usepackage[latin1]{inputenc}
%\usepackage[T1]{fontenc}
%\usepackage{ucs}            % Eventuell benötigt
%\usepackage[utf8x]{inputenc}

%\usepackage{setspace}           % Paket fuer den Zeilenabstand
%\onehalfspacing                 % Setzt den Zeilenabstand auf 1.5

%\usepackage{graphicx}
%\usepackage{listings}
%\usepackage[hyphens]{url}
%%\usepackage{breakurl}
%\usepackage{hyperref}
%\usepackage[usenames]{color}
%\definecolor{light-gray}{gray}{0.90}
%\usepackage[fixlanguage]{babelbib}
%\usepackage{listings}
%%\lstset{numbers=left, numberstyle=\tiny, numbersep=5pt}
%%\lstset{language=Perl}
%\lstloadlanguages{bash,XML,HTML, PHP, Python}
%\selectbiblanguage{german}
%\usepackage{makeidx}
%%\usepackage{pifont}
%\makeindex
%%\usepackage{fancyhdr}
%%\setlength{\headheight}{15.2pt}
%%\pagestyle{fancy}


%\author{Moritz Schächterle, Dominik Heimstädt \& Andrés Cuartas}
%\title{- Studienarbeit Python - \\ WM-Tippspiel 2010 \\}
%\date{11-Dez-2007}

%\pagestyle{myheadings}
%\markright{Schaechterle, Heimstaedt \& Cuartas}
%\lstset{
%	inputencoding=utf8x,
%	extendedchars=\true,
%	language=Python,
%	basicstyle=\tiny,
%	keywordstyle=\bfseries\color{green},
%	identifierstyle=,
	%commentstyle=\color{gray},	
	%stringstyle=\itshape\color{darkred},
%	numbers=left,
%	numberstyle=\tiny,
%	stepnumber=1,
%	breaklines=true,
%	frame=none,
%	showstringspaces=false,
%	tabsize=4,
%	backgroundcolor=\color{light-gray},
%	captionpos=b,
%	float=htbp,
%	frameround=fttt
%}


%%\lstset{language=XML, stringstyle=\ttfamily, tabsize=2, basicstyle=\small, breaklines=true, backgroundcolor=\color{light-gray}, frameround=fttt}
%%              
%% WORKAROUND, damit lstlistoflistings funktioniert: 
%% Quelle: http://www.komascript.de/node/477
%%
%\makeatletter% --> De-TeX-FAQ
%\renewcommand*{\lstlistoflistings}{%
%  \begingroup
%    \if@twocolumn
%      \@restonecoltrue\onecolumn
%    \else
%      \@restonecolfalse
%    \fi
%    \lol@heading
%    \setlength{\parskip}{\z@}%
%    \setlength{\parindent}{\z@}%
%    \setlength{\parfillskip}{\z@ \@plus 1fil}%
%    \@starttoc{lol}%
%    \if@restonecol\twocolumn\fi
%  \endgroup
%}
%\makeatother% --> \makeatletter 

% Diplomarbeit  
% FH Augsburg 
%
%
%
%
%\documentclass[titlepage, 12pt,a4paper]{scrartcl}
\documentclass[titlepage, 12pt, a4paper]{scrreprt}
%\documentclass[titlepage, 12pt, a4paper]{scrbook}
%scrartcl
\usepackage[ngerman]{babel}
%\usepackage[latin1]{inputenc}
\usepackage[T1]{fontenc}
\usepackage{ucs}            % Eventuell benötigt
\usepackage[utf8x]{inputenc}

\usepackage{setspace}           % Paket fuer den Zeilenabstand
\onehalfspacing                 % Setzt den Zeilenabstand auf 1.5

\usepackage{graphicx}
\usepackage{listings}
\usepackage[hyphens]{url}
%\usepackage{breakurl}
\usepackage{hyperref}
\usepackage[usenames]{color}
\definecolor{light-gray}{gray}{0.90}
\usepackage[fixlanguage]{babelbib}
\usepackage{listings}
%\lstset{numbers=left, numberstyle=\tiny, numbersep=5pt}
%\lstset{language=Perl}
\lstloadlanguages{bash,XML,HTML, PHP, Python}
\selectbiblanguage{german}
\usepackage{makeidx}
%\usepackage{pifont}
\makeindex
%\usepackage{fancyhdr}
%\setlength{\headheight}{15.2pt}
%\pagestyle{fancy}


\author{Jorge Andrés Cuartas Monroy}
\title{- Diplomarbeit - \\ Python Frameworks \\}
%\date{11-Dez-2007}

\pagestyle{myheadings}
\markright{Cuartas}
\lstset{
	inputencoding=utf8x,
	extendedchars=\true,
	language=Python,
	basicstyle=\tiny,
	keywordstyle=\bfseries\color{green},
	identifierstyle=,
	%commentstyle=\color{gray},	
	%stringstyle=\itshape\color{darkred},
	numbers=left,
	numberstyle=\tiny,
	stepnumber=1,
	breaklines=true,
	frame=none,
	showstringspaces=false,
	tabsize=4,
	backgroundcolor=\color{light-gray},
	captionpos=b,
	float=htbp,
	frameround=fttt
}


%\lstset{language=XML, stringstyle=\ttfamily, tabsize=2, basicstyle=\small, breaklines=true, backgroundcolor=\color{light-gray}, frameround=fttt}
%              
% WORKAROUND, damit lstlistoflistings funktioniert: 
% Quelle: http://www.komascript.de/node/477
%
\makeatletter% --> De-TeX-FAQ
\renewcommand*{\lstlistoflistings}{%
  \begingroup
    \if@twocolumn
      \@restonecoltrue\onecolumn
    \else
      \@restonecolfalse
    \fi
    \lol@heading
    \setlength{\parskip}{\z@}%
    \setlength{\parindent}{\z@}%
    \setlength{\parfillskip}{\z@ \@plus 1fil}%
    \@starttoc{lol}%
    \if@restonecol\twocolumn\fi
  \endgroup
}
\makeatother% --> \makeatletter 



\begin{document}
\begin{titlepage}
%\vspace*{7cm}
\begin{figure}[htb]
\centering
\includegraphics[width=8cm]{images/hsaugsburglogo.png}
\label{Logo}
\end{figure}

\setlength{\TPHorizModule}{1mm}
\setlength{\TPVertModule}{1mm}


\begin{textblock}{100}(0, 0)
\fontsize{60}{15}
\selectfont
\textbf{Hochschule} 
\end{textblock}

\begin{textblock}{300}(0, 15)
\fontsize{50}{15}
\selectfont
\textbf{Augsburg} University of 
\end{textblock}

\begin{textblock}{100}(60, 30)
\fontsize{50}{15}
\selectfont
Applied Sciences 
\end{textblock}

\begin{textblock}{100}(0, 60)
\fontsize{30}{15}
\selectfont 
\textbf{Diplomarbeit} 
\end{textblock}

\begin{textblock}{200}(0, 70)
\fontsize{20}{15}
\selectfont 
Studienrichtung Informatik 
\end{textblock}

\begin{textblock}{200}(0, 90)
\fontsize{20}{15}
\selectfont 
\textbf{Jorge Andrés Cuartas Monroy}
\end{textblock}

\begin{textblock}{300}(0, 100)
\fontsize{20}{15}
\selectfont 
Python Webframworks\\
und Implemetierung einer Administrationsapplikation\\
zur Verwaltung des Zugangs zu Microcontrollern
\end{textblock}
%\vspace*{1cm}
%\begin{flushleft}
%Hochschule\\
%Augsburg University of\\
%\hspace*{1.6cm}
%Applied Sciences
%\end{flushleft}
%%\enquote{GNUFree Documentation License}.
\end{titlepage}  
%\maketitle
\newpage

\tableofcontents
\newpage
\chapter{Aufgabenstellung}
\section{Einführung}
Im Rahmen der Vorlesung „Internetprogrammierung mit Python“ ist eine
Studienarbeit zu erstellen. Aus mehreren zur Auswahl stehenden Arbeiten ist die
Wahl auf „Fußball-Tippspiel“ gefallen. Die Aufgabe darf im Team bearbeitet
werden mit maximal drei Gruppenmitgliedern. Aus aktuellem Anlass wird die
Anwendug ein Fußball-WM Tippspiel mit der Möglichkeit auf die Begegnungen zu
tippen und für richtige Tipps, Differenz oder Tendenz Punkte zu erhalten. Die
Anwendung soll es darüber hinaus ermöglichen, einzusehen wieviel Punkte der
User momentan hat und auf welchem Platz er steht. Darüberhinaus soll er die
Möglichkeit haben, abgelaufene Tipps anderer Spieler einzusehen.

Als Basis für die Anwendung standen folgende Framework zur Auswahl: Zope,
Turbogears, Pylons und Django in der engeren Auswahl.

Für die Auswahl des geeigneten Frameworks sind folgende Punkte relevant:
arbeiten mit bekannten Techniken/Komponenten wie Python, MySQl,
ORM\footnote{Objekt-Relational Mapping}, HTML, JavaScript und Apache. 
Darüber hinaus sollte die Einarbeitungszeit nicht zu lang sein, da - wie wir aus
Erfahrung wissen - das Einarbeiten in neue Frameworks zeitaufwendig ist. Somit 
ist eine gute Dokumentation ein weiterer wichtiger Punkt für die Auswahl.

Mit allen oben erwähnten Frameworks kommt man sicher schnell und einfach ans
Ziel. Letzendlich fiel die Wahl auf Django, da dieses Framework eine recht
schnelle Entwicklung für unsere Studienarbeit versprach. Unter anderem sticht
die einfache Userverwaltung, der schnell zu konfigurierende Adminbereich, der
zur Verfügung gestellt wird, der OR-Mapper, das Templatesystem und der DRY: 
Don't repeat yourself Ansatz heraus. 

\chapter{Web Framework}
\section{Django}
Das Django Webframework eignet sich für die Erstellung von Webanwendungen. Es
folgt dem MVC\footnote{Model View Control} Muster. Bei Django können die Modelle
der Anwendung, die Objekte mit denen Django arbeitet, mit Hilfe von OR-Mapping
in der Datenbank gespeichert werden. So wird die Datenpersistenz der Anwendung
gewährleistet. Außerdem unterstützt Django mit MySQL, PostgreSQL, Oracle und 
SQLite die wichtigsten Datenbanken. Für die View sind Templates zuständig, die 
mit Hilfe einer eigenen Templatesprache konfiguriert werden. Die Schnittstelle 
nach außen zum Server bieten die Views.py an (dabei handelt es sich, um den 
Controller im MVC-Muster!), die die Anwendung steuern. Diese beinhalten die 
Geschäftslogik und dienen als Verbindung zwischen den Modellen und den Templates.

\section{Klassen}
Für die Anwendung stand zuerst die Erstellung der Klassen bzw. der
Datenbankmodelle im Vordergrund, da sie das Fundament unserer Anwendung 
darstellen. Folgende Klassen werden für die Anwendung benötigt:

\begin{figure}[ht]
 \begin{center}
  \includegraphics[scale=0.5]{pictures/klassen.png}
 \end{center}
 \caption{Benötigte Klassen}
 \label{klassen}
\end{figure}

Im Zentrum stehen die Tipps, welche die User abgeben können. Die Tipps setzen
sich zusammen aus den einzelnen Usern (Tipper) und den Spielbegegnungen. In
diesem Fall die Spiele der WM 2010. Weiterhin setzen sich diese aus den
Mannschaften und weiteren Informationen wie Datum des Spiels und den
Ergebnissen zusammen.
\\
\\

\begin{lstlisting}[caption=Modelle in Django]{modelleDjango}
class Tipps(models.Model):
    user = models.ForeignKey(User, unique=False)
    begegnung = models.ForeignKey(Begegnung, unique=False)
    
    toreHeim = models.IntegerField(max_length=2)
    toreGast = models.IntegerField(max_length=2)
    tippDatum = models.DateTimeField()
    
    def __unicode__(self):
        return u'%s %s' % (self.user, self.begegnung)
   
\end{lstlisting}

\section{OR-Mapping}
Mit Hilfe des eingebauten OR-Mappers in Django können die Tabellen aus den
vorher definierten Klassen als Tabellen in die Datenbank gespeichert werden.
Die Speicherung bzw. Synchronisation geschieht mit Django eigenen Bordmitteln.

\begin{lstlisting}[caption=Datenbanksynchronisation]{sync}
python manage.py syncdb
\end{lstlisting}

Damit Python mit der Datenbank kommunizieren kann, muss vorher ein
Datenbank-Connector installiert werden. Die Installation des Connectors ist 
abhängig von der Datenbank, die benutzt werden soll. Getestet wurden MySQL und 
PostgreSQL. Bei der lokalen Entwicklung unter Ubuntu MySQL und auf dem 
Produktionsserver PostgreSQL. Bei der Speicherung und dem Auslesen der Daten 
erwies sich MySQL toleranter. Für den Produktionsserver mussten drei 
Views\footnote{Django funktion} für PostgreSQL angepasst werden, weil
Exceptions von Django zurückgegeben wurden, die auf Probleme mit dem Speichern 
und Auslesen von Daten zurückzuführen waren.

\section{Administration}
Mit Hilfe der Django eigenen Administrationsoberfläche, die einfach als
Applikation in der \emph{settings.py}\footnote{Datei beinhaltet alle
Umgebungsvariablen des Projektes} installiert werden kann, können leicht die
benötigten Einträge in die Datenbank geschrieben werden. Auch
\emph{Datetime-Felder} werden erkannt und es werden spezielle Widgets für die
Eintragung von Datenfeldern zur Verfügung gestellt. In dem Zusammenhang wurde
entschieden, die benötigten Informationen über Skripte, automatisch befüllen zu
lassen. Genaueres im Kapitel \ref{hilfsskripte} auf Seite
\pageref{hilfsskripte}.

\chapter{Implementierung}
\section{Views}
Die Geschäftslogik wird mit Hilfe der Views erstellt. Der Name ist hier etwas
irreführend, weil man View mit der Ausgabe (dem View im MVC-Muster, das wir
hier "Template" nennen), also HTML verküpft. Die View ist aber die
Schnittstelle durch die die Anwendung mit dem Server/Client kommuniziert. Im 
wmTippspiel wird beim Aufruf der Startseite automatisch die dafür zuständige 
View angesprochen. Diese führt Befehle aus, die ihre Ergenisse einem Template 
übergeben können. Das Template generiert aus den übergebenen Informationen dann
schließlich HTML, welches dem Server zur Weiterleitung an den Client zur 
Verfügung gestellt wird.

\section{URL-Dispatching}
In Django gibt es eine Kontrollinstanz, die es ermöglicht, abhängig von der URL,
auf die verschiedenen Views der Seite zu verweisen. Einstellungen können in der
\emph{urls.py} gemacht werden. Mit Hilfe regulärer Audrücke wird die URL
untersucht, der erste Ausdruck, auf den die URL passt, ermittelt und
die dahinterliegende View ausgeführt.

\begin{lstlisting}[caption=Auszug ursl.py]{urls.py}
urlpatterns = patterns('',
    (r'^$', index ),
    (r'^time/$', current_datetime ),
    (r'^displaymeta/$', display_meta ),
    #(r'^login/$', mylogin ),
    (r'^accounts/logout/$', 'django.contrib.auth.views.logout', {'next_page':'/'}),
    (r'^accounts/login/$', 'django.contrib.auth.views.login'),
    # Example:
    # (r'^wmTippspiel/', include('wmTippspiel.foo.urls')),

    # Uncomment the admin/doc line below and add 'django.contrib.admindocs'
    # to INSTALLED_APPS to enable admin documentation:
    # (r'^admin/doc/', include('django.contrib.admindocs.urls')),

    # Uncomment the next line to enable the admin:
    (r'^admin/', include(admin.site.urls)),
    
    (r'^tippspiel/', include('wmTippspiel.appWMTippspiel.urls')),
    
)
\end{lstlisting}

\section{Django App}
Im wmTippspiel wurde eine App\footnote{Eigenständiger Teil einer Webanwendung,
kann in verschiedenen Projekten eingebunden werden} erstellt und ausgelagert. So
kann die erstellte Applikation auch in anderen Internetseiten ohne viel Aufwand 
eingebunden werden.

\begin{lstlisting}
(r'^tippspiel/', include('wmTippspiel.appWMTippspiel.urls')),
\end{lstlisting}

Die appWMTippspiel bildet den Kern des wmTippspiel-Projekts. Daneben sind
weitere Applikationen, wie eine Registrierungs-Applikation
(siehe Kapitel \ref{registrierung}) denkbar, damit sich User bequem auf der
Seite registrieren können. Diese Applikation kann thematisch gut von der 
appWMTippspiel abgegrenzt und in anderen Projekten eingesetzt werden. Was zu
dem Don't repeat Yourself Gedankten Djangos passt.

\chapter{Zugangsmechanismus}
\section{User-Authentifikation}
In unserem Projekt stellte sich das User-Authentifikation-System als eines der
drei größeren UseCases heraus. Wichtig erschien uns, dass der Besucher beim 
ersten Besuch ohne großen Zeitaufwand einen Account erstellen und dann 
gleich mit dem Tippen loslegen kann. Also eine Registrierung mit automatischem 
Login. Später, bei jedem weiteren Besuch, erfolgt dann der gewohnte Login per 
Username und Passwort. Da die Registrierung, auch aufgrund unseres Layouts, 
nicht mit einem zweiten Passwortkontrollfeld ausgestattet ist, musste
zusätzlich noch eine „Passwort vergessen?“-Funktionalität eingebaut werden.

Die Arbeit mit dem Web-Framework Django ermöglichte nun den Einsatz des 
Framework eigenen Authentifizierungs-Werkzeuges, zu dem hier zunächst ein
kleiner Einblick der Arbeitsweise gegeben werden soll. Django setzt dabei
auf eine User-Session mit Cookie zur Speicherung der Session-ID. Ein Verfahren,
das mittlerweile bei so ziemlich jedem Framework oder CMS Standard ist. Als 
Aufgabenfelder dient die Verwaltung von Usern, Gruppen und deren
Berechtigungen. Da wir in unserem Projekt nur die Gruppierungen der normalen 
User und Superuser (nur für auto-generierten Admin-Bereich) benutzen und selbst
keine eigenen Gruppen anlegen, wird hier nur das User-Model vorgestellt. Im
Grunde genommen benötigen wir für dieses nur die User-Daten Username, Passwort,
Emailadresse und ein paar versteckte Daten, wie „is\_active“, „last\_login“,
etc. So gesehen, reicht uns also das Django eigene User-Model völlig aus und 
kann gleich so übernommen werden. Auch bei den Berechtigungen kommen wir relativ 
spartanisch zurecht, weil wir nur zwischen Inhalten unterscheiden, die der 
eingeloggte User sehen darf und solche, die jeder Besucher sehen darf 
(Admin-Bereich ist außen vor!). Hier reicht also auch eine einfache
Prüfung aus, welche Django selbst in der View-Datei per Decorator übernimmt:

\begin{lstlisting}[caption=Decorator]{Login decorator}
@login_required
    def meinView(request):
    bspVar = 1 …
\end{lstlisting}

\section{Admin-Bereich für das User-Management}
Zuallererst eine gute Nachricht: das Entwickeln des Admin-Bereichs (eine
aufwendige Aufgabe, deren Mühen der normale Besucher nicht würdigt) kann man 
sich mit Django getrost sparen. Es generiert automatisch für jedes erstellte 
Model gewünschte Formulare in einem geschützten Admin-Bereich, der nur für die 
oben angesprochenen Superuser zugänglich ist. Einzelne Änderungen der Formulare
oder Bedienbarkeit können mit Hilfe der Django Online-Dokumentation schnell 
erledigt werden, meistens sind sie aber nicht von Nöten.


\section{Login-Werkzeug von Django}
Auch beim Erstellen eines Logins kann man sich im Großen und Ganzen auf die 
Bibliothek django.contrib.auth.views.login verlassen. Dies kann ganz einfach 
direkt in der url.py aufgerufen werden, wahlweise mit einem Template oder auch 
ganz spartanisch ohne. Hat man sich für ein Template entschieden, ist es hier 
möglich, sich die Input-Felder (HTML) per Django's eigener Template-Sprache 
generieren zu lassen. Oder, wie wir es gemacht haben, selbst mit HTML zu 
erstellen. Dies ermöglichte uns einen schnelleren Zugriff auf die Attribute des
Input-Feldes, da wir Änderungen daran aus layouttechnischen Gründen benötigten.
Ein interessantes Feature bietet die „next“-Funktion des Logins, die, als 
GET-Parameter übergeben, dem Besucher besseres Navigieren erlaubt: z.B. möchte 
der nicht eingeloggte User in der Galerie ein Bild betrachten. Dies ist jedoch 
nur authentifizierten Usern gestattet. Deshalb wird ein next-Parameter 
mitgegeben, in welchem die Adresse zu dem Bild gespeichert wird, damit er nach 
erfolgreichem Login auf die gewünschte Seite weitergeleitet wird.


\section{Logout bei Django}
Für den Logout reicht es ebenfalls die vorgegebene Funktion 
django.contrib.auth.views.logout in der urls.py einzustellen, da sie kein 
eigenes Template oder besondere Einstellungen benötigt.


\section{Registrierung}
Für die Registrierung selbst gibt es allerdings keine Hilfen seitens Django, da
sie meist eine sehr individuelle Sache ist, z.B. möchte man bei einem Webshop 
Kontodaten, bei einer Online-Community vielleicht Geburtsdatum und Hobbies usw.
wissen. Auch bieten sich hier besondere Sicherungsmöglichkeiten wie Captchas,
etc. an, was wir allerdings für unser kleines Tippspiel aufgrund der geringen 
Laufzeit nicht benötigten. Da wir uns einen schnellen Einstieg für den 
Teilnehmer wünschten, verzichteten wir außerdem auf eine Aktivierungsemail und 
loggten den User sofort nach der Registrierung ein.

\section{Passwort vergessen}
Da eine erleichterte Registrierung schnell zu fehlerhaften bzw. vergessenen 
Passwörtern führen kann, wurde noch eine „Passwort vergessen“-Funktion 
eingeführt, die dem Teilnehmer nach dem Eintragen seiner Emailadresse, sein 
neues, zufallsgeneriertes Passwort zuschickt. Auch bei der Entwicklung dieser
Funktion kann man sich wieder auf Django verlassen, da es sowohl ein Werkzeug
zum E-mail versenden anbietet, als auch eine Funktion zur Erstellung von 
zufälligen Passwörtern, was jedoch bei uns selbst durch einen kleinen Einzeiler 
programmiert wurde.

\chapter{Hilfsskripte}\label{hilfsskripte}
\section{Auslesen der Mannschaften}
Um die teilnehmenden Mannschaften nicht alle von Hand in die Datenbank
eintragen zu müssen, haben wir ein Skript geschrieben. Dieses liest mit 
BeautifulSoup den Quelltext der Fifa Homepage 
(http://de.fifa.com/worldcup/standings/index.html) ein, die relevanten 
Informationen werden extrahiert und damit die Datenbank gefüllt. Dabei hat sich
herausgestellt, dass BeautifulSoup nicht ganz so flexibel ist, wie es 
wünschenswert gewesen wäre. So sind die Informationen auf der Seite der Fifa 
z.B. in einer Tabelle (HTML table) angeordnet. Dabei wären Selektoren, wie sie 
bei CSS3 vorkommen, brauchbarer gewesen (z.B. 'tr td.l a strong'). So war es 
z.B. nicht möglich eine Tabellen-Zelle ausgehend von der Klasse der 
Tabellen-Zeile zu selektieren. Dies hätte das Ganze dynamischer gemacht und 
wäre auch mit einem leicht veränderten Design von Seiten der Fifa 
zurechtgekommen. Als Notlösung haben wir die Pfade "statisch" aufgebaut, was
das Skript natürlich schlecht wiederverwendbar macht, da man für jede neue 
Seite die Pfade komplett neu aufbauen muss. Eine Alternative hierbei wäre lxml 
gewesen. Dort wird CSS3-Selektor Unterstützung versprochen. Dagegen ist hier 
problematisch, dass lxml ein Pythonic binding für die Bibliotheken "libxml2"
und  "libxslt" ist. Diese Bibliotheken laufen allerdings unter C und weil wir 
soweit wie möglich "pure Python" bleiben wollten, haben wir uns (auch der 
Einfachheit halber um nicht noch einen Parser "lernen" zu müssen) für 
BeautifulSoup entschieden. Dieses Skript wurde vor dem offiziellen Start des 
Tippspiels einmal zum Befüllen der Datenbank ausgeführt.

\section{Auslesen der Spiel-Ergebnisse}
Das Auslesen der Mannschaften per Skript zu realisieren, welches ja nur einmal 
benutzt wird, war nur eine kleine Erleichterung. Da wir aber nun schon ein
Skript haben, das mit der Struktur der Fifa Seite vertraut ist, haben wir
dieses angepasst, um die Ergebnisse der Spiele auszulesen  (von
http://de.fifa.com/worldcup/matches/index.html). Interessant hierbei war, dass 
bei manchen Spielen ein Halbzeitergebnis angegeben wurde und bei anderen nicht.
Dieses stand allerdings in der gleichen Tabellenzeile wie das "normale" 
Ergebnis. Mit Hilfe von string.find() und anderen String-Funktionen konnten wir
dies jedoch relativ zuverlässig filtern. Zur Identifikation der Spiele zogen
wir das Datum samt Uhrzeit heran, was allerdings zu einem Problem wurde, als 
mehrere Spiele zur gleichen Zeit stattfanden. Hier musste noch auf die 
jeweiligen Mannschaften als zusätzliches Identifikations-Kriterium ausgewichen 
werden. Das grundlegende Problem besteht nunmal darin, die Daten der Homepage 
mit denen der Datenbank abzustimmen und entsprechend zu aktualisieren. Um das 
Skript nicht nach jedem Spiel per Hand ausführen zu müssen, haben wir einen 
Cronjob geschrieben, der das Skript in regelmäßigen Abständen ausführt und so 
die aktuellen Ergebnisse einpflegt.

\chapter{Darstellung der Anwendung}
\section{Django Templatesprache}
Django besitzt eine eigene Template-Sprache, die es ermöglichen soll, dass der 
Designer unabhängig vom Programmierer mit der Implementierung des Layouts 
beginnen kann. Über eine Schnittstelle übergibt Letzterer die benötigten 
Variablen an das Template, wo sie der Designer positionieren kann. Hier tritt 
ein grundlegendes Problem auf: zum Einen möchte der Designer bei jeder Seite 
nicht das komplette Layout neu einbauen müssen, da er sonst im schlimmsten Fall
bei einer Layoutänderung, z.B. Copyright-Rechtsbestimmungen erneuert, mehrere 
hundert Dateien ändern müsste. Aus diesem Grund kann mit Hilfe der 
Template-Sprache eine Basis-Datei angelegt werden, in der die inhaltlichen 
Blöcke markiert werden. Zum Beispiel:

\begin{lstlisting}[caption=Beispiel]{Beispiel Block}

<h1>Überschrift</h1>

\end{lstlisting}
 
Nun kann man weitere Templates anlegen, die sich mit dieser Basis-Datei 
erweitern, jedoch einen eigenen Content-Block besitzen und somit den alten
Block der Basis-Datei mit dem eigenen Inhalt ersetzen.

Weiter muss eine Template-Sprache dem Nutzer vielfältige 
Formatierungsmöglichkeiten bieten, welche dem Designer den Weg zu einer 
dynamischen Seite ebnet. Ein Hilfsmittel sind die bekannten Tags, wie 
IF-ELSE-Blöcke und Schleifen, die Listen, Dictionaries oder Objekte auslesen. 
Ein weiteres Hilfsmittel sind die Filter, die z.B. prüfen, ob die gewünschte 
Variable existiert bzw. ein None oder False zurückgibt und dies dann mit 
vorgegebenen Defaultwerten ersetzen. Häufig werden auch Tag-Filter, benutzt, um 
HTML-Tags zu beseitigen oder zu verändern.


%\section{Portierung auf Produktionsserver}
\chapter{Fazit}
%\section{Erfahrung und Ausblick}
Das Erstellen des WMTippspiel hat viel Spaß gemacht, hat aber auf der
anderen Seite auch viel Zeit in Anspruch genommen. Aufgrund der Aktualität zur
Weltmeisterschaft war es super zu sehen, wie die User der Anwendung
mitgefiebert haben und der Seite treu geblieben sind. Die Arbeit mit Django ging
reibungslos, nur die Portierung auf dem Produnktionsserver hat etwas Zeit in 
Anspruch genommen. Dies war darauf zurückzuführen, dass es einigen Aufwand von Nöten ist, Apache und WSGI mit der
Anwendung kommunizieren zu lassen. Aber nach anfänglichen Hürden war das 
Projekt pünktlich zum Start der Fussball-WM 2010 online.

Wir haben noch viele Ideen für die Anwendung, doch aus Zeitgründen konnten
nicht alle Wünsche umgesetzt werden. Die Ideen nehmen wir mit und vielleicht
entsteht aus dem Projekt ein Bundesliga-Tippspiel für die nächste Saison. 



\begin{thebibliography}{99}
\bibitem{Fifa}
http://de.fifa.com/index.html
\bibitem{BeautifulSoup}
http://www.crummy.com/software/BeautifulSoup/
\bibitem{Universal Feed Parser}
http://www.feedparser.org/
\bibitem{Python}
http://python.org/
\bibitem{Django}
http://www.djangoproject.com/
\bibitem{jQuery}
http://jquery.com/
\end{thebibliography}



\end{document}