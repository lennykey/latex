% Ausarbeitung SAJ 
% FH Augsburg 
%
%
%
%
\documentclass[titlepage, 12pt,a4paper]{scrartcl}
%scrartcl
\usepackage[ngerman]{babel}
%\usepackage[latin1]{inputenc}
\usepackage[T1]{fontenc}
\usepackage{ucs}            % Eventuell benötigt
\usepackage[utf8x]{inputenc}

\usepackage{setspace}           % Paket fuer den Zeilenabstand
\onehalfspacing                 % Setzt den Zeilenabstand auf 1.5

\usepackage{graphicx}
\usepackage{listings}
\usepackage[hyphens]{url}
%\usepackage{breakurl}
\usepackage{hyperref}
\usepackage[usenames]{color}
\definecolor{light-gray}{gray}{0.90}
\usepackage[fixlanguage]{babelbib}
\usepackage{listings}
%\lstset{numbers=left, numberstyle=\tiny, numbersep=5pt}
%\lstset{language=Perl}
\lstloadlanguages{bash,XML,HTML, PHP, Python}
\selectbiblanguage{german}
\usepackage{makeidx}
%\usepackage{pifont}
\makeindex
%\usepackage{fancyhdr}
%\setlength{\headheight}{15.2pt}
%\pagestyle{fancy}


\author{Jorge Andrés Cuartas Monroy}
\title{- Diplomarbeit - \\ Python Frameworks \\}
%\date{11-Dez-2007}

\pagestyle{myheadings}
\markright{Cuartas}
\lstset{
	inputencoding=utf8x,
	extendedchars=\true,
	language=Python,
	basicstyle=\tiny,
	keywordstyle=\bfseries\color{green},
	identifierstyle=,
	%commentstyle=\color{gray},	
	%stringstyle=\itshape\color{darkred},
	numbers=left,
	numberstyle=\tiny,
	stepnumber=1,
	breaklines=true,
	frame=none,
	showstringspaces=false,
	tabsize=4,
	backgroundcolor=\color{light-gray},
	captionpos=b,
	float=htbp,
	frameround=fttt
}


%\lstset{language=XML, stringstyle=\ttfamily, tabsize=2, basicstyle=\small, breaklines=true, backgroundcolor=\color{light-gray}, frameround=fttt}
%              
% WORKAROUND, damit lstlistoflistings funktioniert: 
% Quelle: http://www.komascript.de/node/477
%
\makeatletter% --> De-TeX-FAQ
\renewcommand*{\lstlistoflistings}{%
  \begingroup
    \if@twocolumn
      \@restonecoltrue\onecolumn
    \else
      \@restonecolfalse
    \fi
    \lol@heading
    \setlength{\parskip}{\z@}%
    \setlength{\parindent}{\z@}%
    \setlength{\parfillskip}{\z@ \@plus 1fil}%
    \@starttoc{lol}%
    \if@restonecol\twocolumn\fi
  \endgroup
}
\makeatother% --> \makeatletter 

\begin{document}

\maketitle
\newpage

\section{Webframeworks}
Webframeworks sind im Grunde eine Sammlung von Funktionalität, die es dem
Webentwickler erlaubt Anwendungen zu schreiben. Die Funktionalität besteht aus
verschiedenen Libraries, die den Zugriff auf Datenbanken, die Erstellung von
Templates für die Ausgabe in HMTL oder die Erstellung von Klassen und methoden
für die Geschäftslogik. 

Webanwendungen, die sich auf Datenbanken stützen können mit Webframeworks
relativ schnell erstellt werden, dabei helfen OR-Mapper\footnote{Objekt
Relationale Mapper}, die aus Klassenkonstrukte SQL\footnote{Structed Query
Language} für eine ausgewählte Datenbank erstellen. Die Mapper bieten
darüberhinaus eine eigene Sprache, um auf die Klassen bzw. Informationen in der
Datenbank zuzugreifen. Der Vorteil liegt darin, dass der Code unabhängig von
einer Datenbank erstellt werden kann. Die Geschäftslogik kann abhängig von vom
Framework in einer beliebigen Programmiersprache geschrieben werden, meistens
Java, PHP oder Python. Um die Ausgabe zu generieren bringen die Frameworks eine
Templatesprache mit. Diese ermöglicht die einfache erstellung von HTML aufgrund
Mischung von HTML und Templatecode. Dieser Code dient überwiegend zur
Darstellung von Information, die aus der Geschäftslogik stammt. 

Frameworks werden hauptsächlich unterschieden zwischen Fullstack- und
Glueframeworks.

Ein Fullstackframework bringt alle möglichen Komponenten mit,
um eine Webanwendung schreiben zu können. Meistens sind das ein OR-Mapper für
die Datenbank, eine eigene Templatesprache oder ein
URL-Dispatcher\footnote{http://www.w3.org/Provider/Style/URI} für lesbare Links. Der Vorteil diese Frameworks ist, dass eine klare Linie vorgegeben wird,
um eine Anwendung schreiben zu können. Dies führt meistens zu schnellen
Ergebnissen. 

Auf der anderen Seite gibt es Frameworks, die aus bereits bestehenden
Komponenten bestehen. Dabei sind diese oft lose gekoppelt und können leicht
ausgetauscht werden. So können für diese Frameworks immer die neuesten
Komponenten benutzt werden, die von Gluecode\footnote{Code der Komponente
zusammenbringt} zusammengehalten werden. Es ist aber ein größerer Aufwand nötig,
um diese losen Teile zusammenzubringen. 

 
\section{Gegenüberstellung große u. kleine Frameworks}
\section{Anzupassende Anwendungen}
\section{Framework 1}
\section{Ergebnisse / Übersicht }


\end{document}
