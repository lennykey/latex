\chapter{Anwendung}
\section{Überblick}
Im Zuge der Diplomarbeit soll neben einem Vergleich zu Pythonwebframeworks auch
eine Anwendung für des Rechnertechniklabor der HS Augsburg geschrieben werden.
Dabei soll der Zugriff auf Mikrocontroller ermöglicht werden, in Form einer
Webanwendung. Die Anwendung soll den Benutzern den Zugang auf die
Mikrocontroller ermöglichen. Dabei sollen sie die einzelnen Boards reservieren
können und aus Performance Gründen nur jeweils einem einzelnen Benutzer
gleichzeitig den Zugang auf ein Board ermöglicht werden. Darüber hinaus wird
folgende Funktionalität von der Andwendung aus zugänglich:

\begin{itemize}
  \item Power on/off
  \item Reset
  \item Statusinformationen abrufen
  \item Reservierung
  \item Webupload der Root-Filesysteme
\end{itemize}

Weiterhin bekommt der Benuzter die Möglichkeite Statusinformationen abzurufen, 
die auf der Weboberfläche angezeigt werden. Darüber hinaus auch ein Webupload
von Rootfilesystemen möglich, die auf dem Board mit Hilfe von Skripten auf die
Boards automatisch aufgespielt werden. Zu den Arbeiten gehört weiterhin die
Installation entsprechender Frameworks und nötige Software für den
Reibungslosen Ablauf der Applikation. Die Anwendung wird gegebenenfalls unter
verschiedenen Technologien implementieret. Nach der Reservierung der Boards
bekommt der Benuzter einen Zugang mit Passwort, damit dieser über SSH auf das
Board zugreifen kann. Das Passwort wir nach Reservierung jeweils neu erstellt
und nach dem Abmelden aus der Userverwaltung oder nachdem der User nicht mehr
auf dem Board arbeiten will der Zugang gesperrt, bis ein anderer das Board
reserviert. 

Für jedes der hier vorgestellten Webframeworks wird diese Anwendung entwickelt.
Es soll nach dem Durchlesen dieser Arbeit möglich sein, ein Vergleich zu ziehen,
welches Framework die beste Unterstützung bietet, um Webprojekte im
Pythomumfeld zu erstellen.
