\chapter*{Creative Commons}
\section*{Creative Commons}
\subsection*{Sie dürfen:}
\framebox[30in]{
das Werk bzw. den Inhalt vervielfältigen, verbreiten und öffentlich zugänglich machen
}

I can create basic boxes for text \makebox[3in]{like this}. Notice
that there's a 2in wide space with `like this' in the middle of it.

If I want to put a box around the text, I can use a frame box. The
result looks \framebox[2in]{like this}.

I can also justify the text to the right within a box
\makebox[1.5in][r]{like so} or \framebox[2.5in][l]{like so}.

We can also use quick versions of these. We can just \mbox{do this}
or \fbox{this} to create a quick box that's exactly the size of what we put in it.




\subsection*{Zu den folgenden Bedingungen:}
Namensnennung — Sie müssen den Namen des Autors/Rechteinhabers in der von ihm festgelegten Weise nennen.

Keine kommerzielle Nutzung — Dieses Werk bzw. dieser Inhalt darf nicht für kommerzielle Zwecke verwendet werden.

Keine Bearbeitung — Dieses Werk bzw. dieser Inhalt darf nicht bearbeitet, abgewandelt oder in anderer Weise verändert werden.

\subsection*{Wobei gilt:}
    *   Verzichtserklärung  — Jede der vorgenannten Bedingungen kann aufgehoben werden, sofern Sie die ausdrückliche Einwilligung des Rechteinhabers dazu erhalten.
    * Sonstige Rechte — Die Lizenz hat keinerlei Einfluss auf die folgenden Rechte:
          o Die gesetzlichen Schranken des Urheberrechts und sonstigen Befugnisse zur privaten Nutzung;
          o Das Urheberpersönlichkeitsrecht des Rechteinhabers;
          o Rechte anderer Personen, entweder am Lizenzgegenstand selber oder bezüglich seiner Verwendung, zum Beispiel Persönlichkeitsrechte abgebildeter Personen.
    * Hinweis — Im Falle einer Verbreitung müssen Sie anderen alle Lizenzbedingungen mitteilen, die für dieses Werk gelten. Am einfachsten ist es, an entsprechender Stelle einen Link auf diese Seite einzubinden.

