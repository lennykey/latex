\chapter{Webframeworks}
\section{Definition}
Webframeworks sind im Grunde eine Sammlung von Funktionalität, die es dem
Webentwickler erlaubt Anwendungen zu schreiben. Die Funktionalität besteht aus
verschiedenen Libraries, die den Zugriff auf Datenbanken, die Erstellung von
Templates für die Ausgabe in HMTL oder die Erstellung von Klassen und methoden
für die Geschäftslogik. 

Webanwendungen, die sich auf Datenbanken stützen können mit Webframeworks
relativ schnell erstellt werden, dabei helfen OR-Mapper\footnote{Objekt
Relationale Mapper}, die aus Klassenkonstrukte SQL\footnote{Structed Query
Language} für eine ausgewählte Datenbank erstellen. Die Mapper bieten
darüberhinaus eine eigene Sprache, um auf die Klassen bzw. Informationen in der
Datenbank zuzugreifen. Der Vorteil liegt darin, dass der Code unabhängig von
einer Datenbank erstellt werden kann. Die Geschäftslogik kann abhängig vom
Framework in einer beliebigen Programmiersprache geschrieben werden, meistens
Java, PHP oder Python. Um die Ausgabe zu generieren bringen die Frameworks eine
Templatesprache mit. Diese ermöglicht die einfache erstellung von HTML aufgrund
der Mischung von HTML und Templatecode. Dieser Code dient überwiegend zur
Darstellung von Information, die aus der Geschäftslogik stammt. 

\section{Warum Python}
Die Programmiersprache Python ist eine sehr flexible Programmiersprache. Sie
kann in verschiedenen Gebieten eingesetzt werden. Unter anderem bei der
Verarbeitung von XML, zur Verbindung von Datenbanken, in der Netzwerk 
Programmierung, im Webbereich auch die Erstellung von Programmen mit grafischer
Oberfläche (wxPython, PyQt) ist möglich. In dieser Arbeit wirdvor  allem der 
Webbereich genauer analysiert und verschiedene Webframeworks am Beispiel einer
Anwendung verglichen.

Python ist nicht an ein bestimmtes Programmierparadigma gebunden und je nach
Aufgabe kann ein Paradigma, wie z.B. Objektorientierung, Funktianale
Programmierung oder ein aspektorientierter Ansatz gewählt werden. 

Im Jahre 1991 wurde die Sprache veröffentlicht. Sie wurde im \emph{Centrum voor
Wiskunde en Informatica} in Amsterdam entwickelt. Dabei gilt \emph{Guido van
Rossum} als Erfinder der Sprache. Python wurde unter anderem von C, LISP und
Smalltalk beeinflusst. Wobei Python selbst wiederum Ruby, Boo oder Groovy
beeinflusst hat.

Ja, Python ist gewachsen und hat sich zu einer ausgereiften Sprache samt
eigenem Ökosystem entwickelt.\cite{Python08}

Im Vergleich zu z.B. PHP\cite{phpwebframeworks} gibt es zwar auch viele
Python-Webframeworks\cite{pythonwebframeworks}, diese sind aber weniger
bekannt, mit der Ausnahme von z.B. Django oder TurboGears. Es werden diese zwei
beim Vergleich eingesetzt, zusätlich auch neuere oder weniger bekannte
Frameworks im Pythonökoystem.


\section{Fullstack vs. Glue}
Frameworks werden hauptsächlich unterschieden zwischen Fullstack- und
Glueframeworks.

Ein Fullstackframework bringt alle möglichen Komponenten mit,
um eine Webanwendung schreiben zu können. Unter anderem sind das ein OR-Mapper
für die Datenbank, eine eigene Templatesprache oder ein
URL-Dispatcher\footnote{http://www.w3.org/Provider/Style/URI} für lesbare
Links. Der Vorteil eines solchen Frameworks ist, dass eine klare Linie 
vorgegeben wird, um eine Anwendung schreiben zu können. Dies führt in der Regel
zu schnellen Ergebnissen. Vertreter dieser Art sind z.B. Django.

Auf der anderen Seite gibt es Frameworks, die aus einzelnen
unabhängigen Komponenten bestehen. Dabei sind diese oft lose gekoppelt und
können leicht ausgetauscht werden. So können für diese Frameworks immer die 
neuesten Komponenten benutzt werden, die von
Gluecode\footnote{\url{http://en.wikipedia.org/wiki/Glue_code}} zusammengehalten
werden. Es ist aber ein größerer Aufwand nötig, um diese losen Teile 
zusammenzubringen. Das bekannteste Framework dieser Art in Python ist TurboGears.

\section{Weitere Unterscheidungsmerkmale}

\section{Webframeworks anderer Programmiersprachen }
\section{Arbeiten ohne Framework}
Die Erstellung einer Anwendung ohne ein entsprechendes Framework ist sehr
mühsam. Der Entwlickler hat zwar alle Möglichkeiten eine entsprechende
Anwendung zu schreiben. Ihm wird aber vieles abgenommen. Viele Frameworks
bringen Unterstützung in Domänen, wie z.B. Sicherheit, Benutzerverwaltung und
Applikationsadministration. Diese müssen, falls gewünscht neu geschribeben
werden und lenke somit von der eigentlichen Applikation ab. Wenn also eine
Applikation eine gewissen Komplexität erreicht hat oder erreichen soll, ist es
ratsam sich mit einem Frameworkt auseinander zu setzen und von erfahrenen
Entwicklern, die diese Arbeit bereits geleistet haben zu profitieren. Die
meisten Frameworks arbeiten nach dem Prinzip „Don't repeat your self“ oder
„KISS“ keep it simple stupid. Somit werden immer wiederkehrende Aufgaben, die
mit komplexen Projekten verbunden sind bereits vorgefertigt geliefert und der
Entwlickler kann sich sofort dem eigentlichen Projekt widmen. 


