\chapter{Webframeworks}
\section{Definition}
Webframeworks sind im Grunde eine Sammlung von Funktionalität, die es dem
Webentwickler erlaubt Anwendungen zu schreiben. Die Funktionalität besteht aus
verschiedenen Libraries, die den Zugriff auf Datenbanken, die Erstellung von
Templates für die Ausgabe in HMTL oder die Erstellung von Klassen und methoden
für die Geschäftslogik. 

Webanwendungen, die sich auf Datenbanken stützen können mit Webframeworks
relativ schnell erstellt werden, dabei helfen OR-Mapper\footnote{Objekt
Relationale Mapper}, die aus Klassenkonstrukte SQL\footnote{Structed Query
Language} für eine ausgewählte Datenbank erstellen. Die Mapper bieten
darüberhinaus eine eigene Sprache, um auf die Klassen bzw. Informationen in der
Datenbank zuzugreifen. Der Vorteil liegt darin, dass der Code unabhängig von
einer Datenbank erstellt werden kann. Die Geschäftslogik kann abhängig vom
Framework in einer beliebigen Programmiersprache geschrieben werden, meistens
Java, PHP oder Python. Um die Ausgabe zu generieren bringen die Frameworks eine
Templatesprache mit. Diese ermöglicht die einfache erstellung von HTML aufgrund
der Mischung von HTML und Templatecode. Dieser Code dient überwiegend zur
Darstellung von Information, die aus der Geschäftslogik stammt. 

\section{Warum Python}
Die Programmiersprache Python ist eine sehr flexible Programmiersprache. Sie
kann in verschiedenen Gebieten eingesetzt werden. Unter anderem bei der
Verarbeitung von XML, zur Verbindung von Datenbanken, in der Netzwerk 
Programmierung, im Webbereich auch die Erstellung von Programmen mit grafischer
Oberfläche (wxPython, PyQt) ist möglich. In dieser Arbeit wirdvor  allem der 
Webbereich genauer analysiert und verschiedene Webframeworks am Beispiel einer
Anwendung verglichen.

Python ist nicht an ein bestimmtes Programmierparadigma gebunden und je nach
Aufgabe kann ein Paradigma, wie z.B. Objektorientierung, Funktianale
Programmierung oder ein aspektorientierter Ansatz gewählt werden. 

Im Jahre 1991 wurde die Sprache veröffentlicht. Sie wurde im \emph{Centrum voor
Wiskunde en Informatica} in Amsterdam entwickelt. Dabei gilt \emph{Guido van
Rossum} als Erfinder der Sprache. Python wurde unter anderem von C, LISP und
Smalltalk beeinflusst. Wobei Python selbst wiederum Ruby, Boo oder Groovy
beeinflusst hat.

Ja, Python ist gewachsen und hat sich zu einer ausgereiften Sprache samt
eigenem Ökosystem entwickelt.\cite{Python08}

Im Vergleich zu z.B. PHP\cite{phpwebframeworks} gibt es zwar auch viele
Python-Webframeworks\cite{pythonwebframeworks}, diese sind aber weniger
bekannt, mit der Ausnahme von z.B. Django oder TurboGears. Es werden diese zwei
beim Vergleich eingesetzt, zusätlich auch neuere oder weniger bekannte
Frameworks im Pythonökoystem.


\section{Fullstack vs. Glue}
Frameworks werden hauptsächlich unterschieden zwischen Fullstack- und
Glueframeworks.

Ein Fullstackframework bringt alle möglichen Komponenten mit,
um eine Webanwendung schreiben zu können. Unter anderem sind das ein OR-Mapper
für die Datenbank, eine eigene Templatesprache oder ein
URL-Dispatcher\footnote{\url{http://www.w3.org/Provider/Style/URI}} für lesbare
Links. Der Vorteil eines solchen Frameworks ist, dass eine klare Linie 
vorgegeben wird, um eine Anwendung schreiben zu können. Dies führt in der Regel
zu schnellen Ergebnissen. Vertreter dieser Art sind z.B. Django.

Auf der anderen Seite gibt es Frameworks, die aus einzelnen
unabhängigen Komponenten bestehen. Dabei sind diese oft lose gekoppelt und
können leicht ausgetauscht werden. So können für diese Frameworks immer die 
neuesten Komponenten benutzt werden, die von
Gluecode\footnote{\url{http://en.wikipedia.org/wiki/Glue_code}} zusammengehalten
werden. Es ist aber ein größerer Aufwand nötig, um diese losen Teile 
zusammenzubringen. Das bekannteste Framework dieser Art in Python ist TurboGears.

\section{Weitere Unterscheidungsmerkmale}
\subsection{Funktionalität}
Wenn ein neues Produkt das Interesse geweckt hat, liegt es meistens an der
Funktionalität, die es anbietet. Für eine Webapplikation ist es wichtigt, dass
auf eine Art und Weise der Client auf die Datein mittels Browser zugreifen
kann. Als Schnittstelle bietet sich im Pythonumfeld
WSGI\footnote{\url{http://wsgi.org/wsgi/}} an, damit der Webserver mit der
Webapplikation kommunizieren kann, die wiederum HTML an den Client liefert. Für
einfache Anwendungen reicht es aus HTML-Code mit Hilfe von Python zu generieren
und diesem dem User zur Verfügung zu stellen. Soll die Anwendung aber
komplexere Ausmaße haben und zum Beispiel eine Userverwaltung mit LogIn
Möglichkeit bieten ist es ratsam ein Framework auszuwählen, welches diese
Funktionalität mit einfachen Handgriffen bietet. Auch die Kommunikation zu
Datenbanken ist ein wichtiger Bestandteil, da viele Webapplikationen Klassen
bereitstellen, mittles derer Objekte generiert werden, die miteinander
interagieren. Daraus entstehen meinstens Informationen die persistent gehalten
werden müssen. Beispiele sind Webshops, Blogs/Wikis oder soziale Netzwerke. Bei
der Entwicklung sollte auch berücksichtigt werden, dass Webdesigner leicht im
Entwicklungsprozess intergriert werden können. Dabei sollte wichtig sein, dass
CSS und JavaScript (oder JavaScript-Frameworks), die als Standard bei der
Gestaltung gelten eingesetzt werden können. Dabei spielen Templatesprachen für
die erstellung des HTML-Codes eine große Rolle, denn viele Webframeworks bieten
Templatesprachen, die sich an Pythoncode anlehnen aber nur geringe logische
Funktionalität bieten und im Grunde nur für die Darstellung der einzelnen
Seiten konzipiert sind.

\subsection{Dokumentation}
Viele Projekte und im Spezeiellen hier Webframeworks werden hauptsächlich wegen
ihre Funktionalität ausgewählt. Ein weiterer Grund kann auch ihre Popularität
sein auf Grund eines Hypes. Ein sehr wichtiger Aspekt, um überhaupt mit einem
Framework oder Programmiersprache anzufangen, ist die Lernhürde die manchmal
hoch sein kann. Deswegen ist es wichtig und sollte eine Priorität der
Entwickler sein die Frameworks sehr gut zu dokumentieren und Anfängertutorials
zu anzubieten. 

\section{Bekannte Webframeworks im Pythonumfeld}
\subsection{Django}
Django ist ein High-Level Python Webframework, das eine schnelle Entwicklung
fördert und ein ein saubers und pragmatischen Design verfolgt. 

Django wurde designt, um zwei Herauforderungen zu meistern: Intensiver
Deadlines eines Newsrooms und den strengen Anforderungen der erfahrenden
Webentwicklern, die das Framework geschreiben haben, gerecht zu werden. Django
ermöglicht es performante und elegante Anwendungen, schnell zu entwickeln.
\cite{django}


%Django ist ein Fullstackframework, das eine schnelle Entwicklung verspricht und
%mit verschiedenen Datenbanken arbeiten kann. Dies geschieht durch einen eigenen
%OR-Mapper. Für die Darstellung der Webseiten wird eine eigene Django-Template
%engine verwendet, die sich an Python anlehnt.
\subsection{Pylons}
Pylons verbindet verbindet Ideen aus der Welt von Ruby, Python und Perl. Dabei
ist es ein sehr flexibles Python-Webframework und verhalf dem WSGI-Standard zum
Durchbruch. Das Ziel von Pylons ist es, die Webentwicklung schnell, flexibel und
einfach zu gestalten.\cite{pylons}

\subsection{Zope}
Die Zope Softwarebibliothek ermöglicht die komponentenbasierte Entwicklung von 
Web-Anwendungen in der objektorientierten Programmiersprache Python. In der 
komponentenbasierten Programmierung werden komplexe Anwendungen mit Hilfe 
wiederverwendbarer Komponenten erstellt. Eine Komponente stellt dabei  die
Implementierung einer bestimmten genau spezifizierten Funktionalität
dar.\cite{ueberzope}

\subsection{web2py}
web2py wurde inpiriert von Ruby on Rails und, wie Rails, liegt der Focus in
schneller Webentwicklung und folgt dem Model View Controller Muster. web2py
unterscheidet sich von Rails, weil es auf Python basiert. Weil es eine
umfassendes Web basierte Administrationsoberfläche bereitstellt, darüber hinaus
beinhaltet es Bibliotheken, um mit mehr Protokollen zurechtzukommen und ist auf
der Google App Engine lauffähig.\cite{web2py} 


\section{Webframeworks anderer Programmiersprachen}
\subsection{Überblick}
Zur Erstellung von Webapplikationen gibt es viele Möglichkeites dies zu
bewerkstelligen. Der einfachste Ansatze ist, eine Programmiersprache zu lernen,
mit deren Hilfe HTML-Code erzeugt und dem Client zur Verfügung gestellt
wird.

Der erste Kontakt wird sicherlich mit PHP gemacht. Leicht zu erlernen und
mit einer Vielzahl von Funktionen. Darüber hinaus existieren eine Menge
Frameworks, die den Entwlickler bei der Arbeit unterstützen. Unter anderen Zend,
CakePHP oder Symphony.

Andere Programmiersprachen, wie Ruby, Java oder
Perl\footnote{\url{http://www.socialtext.net/perl5/index.cgi?web_frameworks}} 
ermöglichen ebenfalls Programme fürs Web zu schreiben und einem breiten 
Publikum zur Verfügung zu stellen. Darunter bekannte wie Ruby on Rails, Spring 
oder Jifty. Zwar liegt der Focus auf Python, dennoch ist es wichtig andere 
Möglichkeiten der Webentwicklung für einen besseren Überblick kurz vorzustellen.

\subsection{Zend}
Das Zend Framework erweitert die Kunst und den Geist von PHP, dabei besiert es
auf Einfachheit, beste praktische Vorgehensweise, freundliche Lizenzierung für
Firmen und einer gründlich getesteten Codebasis. Dabei liegt der Focus des Zend
Frameworks auf sicherere und moderne Web 2.0 Anwendungen und Webservices.
\cite{zend}

\subsection{Ruby on Rails}
Rub on Rails ist eine OpenSource Webframework für die Zufriedenstellung der
Webentwickler und kontinuierliche Unterstützung der Produktivität. Damit lässt
es sich wunderbaren code schreiben wobei das Design Paradigma Convention over
Configuration\footnote{Convention over Configuration: Dabei ist das Design
Paradigma gemeint, bei dem der Entwickler Regeln folgen muss um z.B. dass
Primärschlüssel in der Datenbank Integer sein müssen oder Klassennamen und
Tabellennamen gleich lauten müssen} angewandt wird.\cite{rubyonrails}

„Ruby on Rails is a breakthrough in lowering the barriers of entry to
programming. Powerful web applications that formerly might have taken weeks or 
months to develop can be produced in a matter of days.“\footnote{Tim O'Reilly,
Founder of O'Reilly Media: \url{http://rubyonrails.org/quotes} }

\subsection{Catalyst}
Das Catalyst ist ein in Perl geschriebenes  MVC\footnote{MVC: Model View
Controler Design Muster
\url{http://www.oracle.com/technetwork/java/mvc-140477.html}} Webframework. 
Dieses Framework kann auf verschiedenen Plattformen installiert werden und ist 
für schnelle Webentwicklung geeignet.\cite{catalystperlnet}

Bei der Entwicklung wird bedacht auch KISS\footnote{KISS: Keep it simple
stupid \url{http://techcrunch.com/2009/04/28/keep-it-simple-stupid/}} gelegt,
was dazu führt, dass Catalyst skalierbar und robust bleibt. Was sich bei den 
daraus entwickelten Applikationen bemerkbar macht.\cite{catalystframework}

\subsection{Wicket}
Mit zweckmäßiger Aufteilung von Darstellung und Logik, ein POJO\footnote{POJO: Plain Old Java
Object \url{http://www.martinfowler.com/bliki/POJO.html}} Datenmodell und
erfrischendem Verzicht von XML, macht Apache Wicket das Entwickeln von
Webapplikationen wieder einfach und angenehm. Es wird mit unnötigem Code,
komplexem Debugging und brüchigem Code gebrochen, um wiederverwendbare
Komponenten, die nur auf Java und HTML aufbauen zu erstellen.\cite{wicket}

\section{Arbeiten ohne Framework}
Die Erstellung einer Anwendung ohne ein entsprechendes Framework ist
natürlich möglich doch oft sehr mühsam. Der Entwlickler hat alle Freiheiten eine
Anwendung zu schreiben, ihm wird aber mit einem guten Framework vieles
erleichtert. Webframeworks bringen Unterstützung in Aspekten, wie z.B.
Sicherheit, Benutzerverwaltung und Applikationsadministration. Diese müssen, 
falls benötigt von Grund auf immer wieder neu geschriben werden und lenken
somit von der eigentlichen Entwicklung der Kernapplikation ab. 

Die meisten Frameworks arbeiten nach dem Prinzip „Don't repeat your self“ oder 
„KISS“ keep it simple stupid. Somit werden immer wiederkehrende Aufgaben, die 
mit komplexen Projekten verbunden sind bereits vorgefertigt geliefert und der 
Entwlickler kann sich sofort der eigentlichen Kernaufgabe widmen ohne
sich um nötigen Boilerplate kümmern zu müssen.


